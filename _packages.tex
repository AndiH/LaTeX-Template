%!TEX root = template.tex

% -- packages -------------------------------------- %

% basic layout of the document
\documentclass[
  11pt, a4paper,        % font and paper size
  fleqn,                % align equations to the left
  twoside, openright,   % two sided option with new chapters on the right page
  \status               % the status (defined in main document)
]{memoir}

% general stuff
\usepackage[utf8]{inputenc}   % Input-Encoding
                              % (encoding which is used for saving)
\usepackage[TS1,T1]{fontenc}  % Output-Encoding
                              % (Latin1 [T1] with additional sepecial characters [TS1])

% \usepackage[ngerman]{babel}   % German language package
\usepackage[english]{babel}   % English language package

% font and symbols
\usepackage[bitstream-charter]{mathdesign} % Bitstream Character font, not-italic greek letter like \alphaup
\usepackage{microtype}        % better type setting
% \usepackage{amssymb,amsmath}  % more math symbols and environments % mathdesign interferes with amssymb
% \usepackage{textcomp}         % even more special characters
\usepackage{tikz}             % used to make circled numbers
\usepackage{color}            % define your own colors
\usepackage{scrhack}          % fixes an error caused by listings package
\usepackage{listings}         % enables syntax highlighting for your code

% references
% \usepackage[backend=biber]{biblatex}

% physics and maths
\usepackage{amsmath}
\usepackage{wasysym}          % glyphs and other symbols, e.g. \int, http://www.ctan.org/pkg/wasysym
\usepackage{siunitx}          % macros for printing numbers (\num{12345}) and units (\SI{1e7}{\kilogram\per\metre\squared}), also interesting: \SIrange{10}{30}{\metre}, \mega\electronvolt, \DeclareSIUnit. \per can be set individually via SI[per-mode=reciprocal], or per-mode-fraction to overwrite the new default in the next line; http://www.ctan.org/pkg/siunitx
\usepackage{commath}          % defines differential operators (\del, \dif)
\usepackage{feynmp}           % Feynman diagrams
\usepackage{cancel, slashed}  % Feynman notation
% \usepackage{maybemath}        % For context-sensitive maths, like headings
\usepackage{hepparticles}     % particle names
\usepackage{hepnames}         % shortcuts for lots of particle names
\usepackage{braket}           % provides \bra and \ket notation

% configuration
\usepackage[pdfencoding=auto]{hyperref}         % hyperlinks in PDF
% The two following packages don't work with the memoir style
% \usepackage{fancyhdr}         % configure your page header
% \usepackage[noindentafter]{titlesec}  % configure titles

% image stuff
\usepackage{graphicx}        % include graphics
\usepackage{rotating}         % rotate images
% \usepackage{caption}          % customize your captions (memoir incompatible)
% \usepackage{subfig}           % sub-captions (memoir incompatible, already included there with \subbottom)
\usepackage{wrapfig}          % wrap text around images and tables

% table stuff
\usepackage{booktabs}         % professional tables
\usepackage{multirow}         % combine table rows
% \usepackage{arydshln}         % dashed lines in tables

% additional packages (mixed)
% \usepackage{keystroke}        % display keystrokes
\usepackage{xspace}           % adds spaces to the end of commands if needed
\usepackage{pdflscape}        % allows for landscaped, pdf-rotated single pages
\usepackage[acronym,nonumberlist,xindy]{glossaries}  % glossaries % xindy is used for sorting non-latin characters correctly
\usepackage{multicol}         % offers multi column environments
\usepackage{ifdraft}          % offers \ifdraft{}{} command to do stuff while in draft mode

% packages that help during draft versions
% (change document option from draft to final to disable)
\usepackage[obeyDraft,textsize=scriptsize]
  {todonotes}                 % notes and missing images
\usepackage{showlabels}       % show label names next to numbering
